%%%%%%%%%%%%%%%%%%%%%%%%%%%%%%%%%%%%%%%%%
% Structured General Purpose Assignment
% LaTeX Template
%
% This template has been downloaded from:
% http://www.latextemplates.com
%
% Original author:
% Ted Pavlic (http://www.tedpavlic.com)
%
% Note:
% The \lipsum[#] commands throughout this template generate dummy text
% to fill the template out. These commands should all be removed when
% writing assignment content.
%
%%%%%%%%%%%%%%%%%%%%%%%%%%%%%%%%%%%%%%%%%

%----------------------------------------------------------------------------------------
%	PACKAGES AND OTHER DOCUMENT CONFIGURATIONS
%----------------------------------------------------------------------------------------

\documentclass{article}
\setcounter{tocdepth}{4}
\setcounter{secnumdepth}{4}
\usepackage{fancyhdr} % Required for custom headers
\usepackage{lastpage} % Required to determine the last page for the footer
\usepackage{extramarks} % Required for headers and footers
\usepackage{graphicx} % Required to insert images
\usepackage{lipsum} % Used for inserting dummy 'Lorem ipsum' text into the template
\usepackage{url}
\usepackage{amsmath}
\usepackage{amssymb}
\usepackage{arydshln}
\usepackage{mathtools}
\usepackage{breqn}

% Margins
\topmargin=-0.45in
\evensidemargin=0in
\oddsidemargin=0in
\textwidth=6.5in
\textheight=9.0in
\headsep=0.25in

\linespread{1.1} % Line spacing

% Set up the header and footer
\pagestyle{fancy}
\lhead{\hmwkClass\ : \hmwkTitle} % Top center header
\rhead{\firstxmark} % Top right header
\lfoot{\lastxmark} % Bottom left footer
\cfoot{} % Bottom center footer
\rfoot{Page\ \thepage\ } % Bottom right footer
\renewcommand\headrulewidth{0.4pt} % Size of the header rule
\renewcommand\footrulewidth{0.4pt} % Size of the footer rule

\setlength\parindent{0pt} % Removes all indentation from paragraphs

%----------------------------------------------------------------------------------------
%	NAME AND CLASS SECTION
%----------------------------------------------------------------------------------------

\newcommand{\hmwkTitle}{Kalman Filtering for an Inverted Pendulum Proposal} % Assignment title
\newcommand{\hmwkDueDate}{Wednesday,\ May\ 9,\ 2018} % Due date
\newcommand{\hmwkClass}{MAE\ 298} % Course/class
\newcommand{\hmwkClassTime}{} % Class/lecture time
\newcommand{\hmwkClassInstructor}{Professor Lin} % Teacher/lecturer
\newcommand{\hmwkAuthorName}{Jordan McCrone, Sarah O'Meara, Peng Chen} % Your name

%----------------------------------------------------------------------------------------
%	TITLE PAGE
%----------------------------------------------------------------------------------------

\title{
\vspace{2in}
\textmd{\textbf{\hmwkClass:\ \hmwkTitle}}\\
\normalsize\vspace{0.1in}\large{Due\ on\ \hmwkDueDate}\\
\vspace{0.1in}\large{\textit{\hmwkClassInstructor\ \hmwkClassTime}}
\vspace{3in}
}

\author{\textbf{\hmwkAuthorName}}
\date{} % Insert date here if you want it to appear below your name

%----------------------------------------------------------------------------------------

\begin{document}

\maketitle

%----------------------------------------------------------------------------------------
%	TABLE OF CONTENTS
%----------------------------------------------------------------------------------------

%\setcounter{tocdepth}{1} % Uncomment this line if you don't want subsections listed in the ToC

\newpage
\tableofcontents
\listoffigures
\listoftables
\newpage

\section{Examples}

Example superscript JACO\textsuperscript{2} Example citation \cite{weisz2017assistive}

\begin{figure}[h!]
	\centering
	\includegraphics[width=5cm,keepaspectratio]{kinova.png}
	\caption{JACO\textsuperscript{2} 6-DOF Manipulator}
	\label{fig:diagram1}
\end{figure}

Example Figure \ref{fig:diagram1} Example symbols $\epsilon$. Example table

\begin{table}[h!]
\centering
\begin{tabular}{ |c |c  |c |c |c  |c  |c |}
\hline
	 i & 1 & 2 & 3 & 4 & 5 & 6 \\ \hline
	 a\textsubscript{i} & 0 & d\textsubscript{2} & 0 & 0 & 0 & 0 \\ \hline
	 $\alpha$\textsubscript{i}  & -90 & 0 & -90 & 90 & -90 & 0 \\ \hline
	 $\theta$\textsubscript{i} & $\theta$\textsubscript{1} & $\theta$\textsubscript{2} & $\theta$\textsubscript{3} & $\theta$\textsubscript{4} & $\theta$\textsubscript{5} & $\theta$\textsubscript{6} \\ \hline
	 s\textsubscript{i} & -d\textsubscript{1} & 0 & $\epsilon$\ &  d\textsubscript{3} + d\textsubscript{4} & 0 & d\textsubscript{5} + d\textsubscript{6} \\ \hline
\end{tabular}
\caption{D-H Parameters}
\label{table:1}
\end{table}

Example matrix with no equation number

\[
A_{1} = 
\begin{bmatrix}
	C\theta_{1} & 0 & -S\theta_{1} & 0 \\
	S\theta_{1} & 0 & C\theta_{1} & 0  \\
	0 & -1 & 0 & -d_{1} \\
	0 & 0 & 0 & 1 \\ 
\end{bmatrix}
\]

\section{Introduction}

\begin{equation}
\textbf{x} = 
\begin{bmatrix}
	x \\
	\dot{x} \\
	\theta \\
	\dot{\theta} \\
\end{bmatrix}
\label{xMatrix}
\end{equation}

\begin{equation}
\beta = \dfrac{4}{3} - \dfrac{m}{M+m}
\end{equation}

\begin{equation}
\textbf{A} = 
\begin{bmatrix}
	1 & \tau & 0 & 0 \\
	0 & 1 & \dfrac{-gm\tau}{(M+m)\beta} & 0 \\
	0 & 0 & 1 & \tau \\
	0 & 0 & \dfrac{g\tau}{l\beta} & 1 \\
\end{bmatrix}
\label{aMatrix}
\end{equation}

\begin{equation}
\textbf{A} = 
\begin{bmatrix}
	1 & \tau & 0 & 0 \\
	0 & 1 & \dfrac{-gm\tau}{(M+m)\beta} & 0 \\
	0 & 0 & 1 & \tau \\
	0 & 0 & \dfrac{g\tau}{l\beta} & 1 \\
\end{bmatrix}
\label{aMatrix}
\end{equation}

\section{References}

%\cite{*}
\bibliography{references}
\bibliographystyle{ieeetr}

\pagebreak
\section{Appendices}

\begin{dmath}
	\hat{z_{6}} \times \vec{r_{6}} = \left[\begin{matrix}0\\0\\0\end{matrix}\right]
\end{dmath}


\end{document}
